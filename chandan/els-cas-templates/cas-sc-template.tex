%% 
%% Copyright 2019-2021 Elsevier Ltd
%% 
%% This file is part of the 'CAS Bundle'.
%% --------------------------------------
%% 
%% It may be distributed under the conditions of the LaTeX Project Public
%% License, either version 1.2 of this license or (at your option) any
%% later version.  The latest version of this license is in
%%    http://www.latex-project.org/lppl.txt
%% and version 1.2 or later is part of all distributions of LaTeX
%% version 1999/12/01 or later.
%% 
%% The list of all files belonging to the 'CAS Bundle' is
%% given in the file `manifest.txt'.
%% 
%% Template article for cas-sc documentclass for 
%% single column output.

\documentclass[a4paper,fleqn]{cas-sc}

% If the frontmatter runs over more than one page
% use the longmktitle option.

%\documentclass[a4paper,fleqn,longmktitle]{cas-sc}

%\usepackage[numbers]{natbib}
%\usepackage[authoryear]{natbib}
\usepackage[authoryear,longnamesfirst]{natbib}
\usepackage{graphicx}
\usepackage{multirow}
\usepackage[normalem]{ulem}
\useunder{\uline}{\ul}{}
\usepackage{lscape}
\usepackage{longtable}
\usepackage{booktabs,tabularx}
\usepackage{float}

%%%Author macros
\def\tsc#1{\csdef{#1}{\textsc{\lowercase{#1}}\xspace}}
\tsc{WGM}
\tsc{QE}
%%%

% Uncomment and use as if needed
%\newtheorem{theorem}{Theorem}
%\newtheorem{lemma}[theorem]{Lemma}
%\newdefinition{rmk}{Remark}
%\newproof{pf}{Proof}
%\newproof{pot}{Proof of Theorem \ref{thm}}

\begin{document}
\let\WriteBookmarks\relax
\def\floatpagepagefraction{1}
\def\textpagefraction{.001}

% Short title
\shorttitle{The Multiview Deep Learning Model explores populous Indian cities.}    

% Short author
\shortauthors{S. Kumar}  

% Main title of the paper
\title [mode = title]{The Multiview Deep Learning Model ventures into some of India's most populated cities}  


\begin{comment}
% Title footnote mark
% eg: \tnotemark[1]
\tnotemark[<tnote number>] 

% Title footnote 1.
% eg: \tnotetext[1]{Title footnote text}
\tnotetext[<tnote number>]{<tnote text>} 

% First author
%
% Options: Use if required
% eg: \author[1,3]{Author Name}[type=editor,
%       style=chinese,
%       auid=000,
%       bioid=1,
%       prefix=Sir,
%       orcid=0000-0000-0000-0000,
%       facebook=<facebook id>,
%       twitter=<twitter id>,
%       linkedin=<linkedin id>,
%       gplus=<gplus id>]

\author[<aff no>]{Subham Kumar}[<options>]

% Corresponding author indication
\cormark[<corr mark no>]

% Footnote of the first author
\fnmark[<footnote mark no>]

% Email id of the first author
\ead{subh700454@gmail.com}

% URL of the first author
\ead[url]{<URL>}

% Credit authorship
% eg: \credit{Conceptualization of this study, Methodology, Software}
\credit{<Credit authorship details>}

% Address/affiliation
\affiliation[<aff no>]{organization={},
            addressline={}, 
            city={},
%          citysep={}, % Uncomment if no comma needed between city and postcode
            postcode={}, 
            state={},
            country={}}

\author[<aff no>]{<author name>}[<options>]

% Footnote of the second author
\fnmark[2]

% Email id of the second author
\ead{}

% URL of the second author
\ead[url]{}

% Credit authorship
\credit{}

% Address/affiliation
\affiliation[<aff no>]{organization={},
            addressline={}, 
            city={},
%          citysep={}, % Uncomment if no comma needed between city and postcode
            postcode={}, 
            state={},
            country={}}

% Corresponding author text
\cortext[1]{Corresponding author}

% Footnote text
\fntext[1]{}

% For a title note without a number/mark
%\nonumnote{}


\end{comment}



% Here goes the abstract
\begin{abstract}
Pollution is a growing worldwide problem that endangers the environment \cite{Fortunato2010}, human health, and biodiversity. It manifests itself in a variety of ways, including air pollution caused by car emissions and industrial operations, water pollution caused by chemical runoff and inappropriate waste disposal, and soil contamination caused by dangerous chemicals and poor agricultural practices. These pollutants not only harm the quality of natural resources but also contribute to climate change and ozone layer depletion. Pollution must be addressed collectively by tough legislation, sustainable practices, and the use of eco-friendly technology. We can protect the earth for future generations and build a cleaner, healthier environment for everybody if we recognize the seriousness of the problem and take responsible action.
Air pollution, water pollution, soil pollution, noise pollution, and light pollution are all examples of pollution. Air pollution is one of the most serious and pressing issues, owing to its extensive influence on human health and the environment. Particulate matter having a diameter of 2.5 micrometres or smaller, generally referred to as $PM_{2.5}$, is very important in terms of air pollution. These microscopic particles may readily enter our respiratory system and penetrate deep into the lungs, producing a variety of health concerns such as respiratory disorders, cardiovascular disease, and even early death. $PM_{2.5}$ is an important aspect in air quality monitoring and mitigation activities since lowering its levels is key for improving public health and the overall health of our planet.
\end{abstract}

% Use if graphical abstract is present
%\begin{graphicalabstract}
%\includegraphics{}
%\end{graphicalabstract}

% Research highlights
\begin{highlights}
\item 
\item 
\item 
\end{highlights}

% Keywords
% Each keyword is seperated by \sep
\begin{keywords}
 \sep Air Pollution \sep  Deep Learning \sep Time Series \sep MultiView
\end{keywords}

\maketitle

% Main text
\section{Introduction}
\begin{equation}
	YL=\sum_{i=1}^{H}\sum_{j=1}^{w}\sum_{k^L=1}^{H} w_{i,jk^L,b*x_{(i^{L+1},j^{L+1}+j,k^L,b)}}
\end{equation}
\begin{equation}
	f*=argmin(\mu (X,y,f),subject of \in F)
\end{equation}

\begin{equation}
	\frac{\partial_z}{\partial_{vec}(\omega^l)}=\phi(X^l)^T*\frac{\partial_z}{\partial_{x^{l+1}}}
\end{equation}

\section*{CRediT authorship contribution statement}
\section*{Data availability}
\section*{Acknowledgments}
\label{}

% Numbered list
% Use the style of numbering in square brackets.
% If nothing is used, default style will be taken.
%\begin{enumerate}[a)%\item 
%\item 
%\item 
%\end{enumerate}  

% Unnumbered list
%\begin{itemize}
%\item 
%\item 
%\item 
%\end{itemize}  

% Description list
%\begin{description}
%\item[]
%\item[] 
%\item[] 
%\end{description}  


\begin{comment}

\begin{table}[<options>]
\caption{}\label{tbl1}
\begin{tabular*}{\tblwidth}{@{}LL@{}}
\toprule
  &  \\ % Table header row
\midrule
 & \\
 & \\
 & \\
 & \\
\bottomrule
\end{tabular*}
\end{table}



\end{comment}
% Uncomment and use as the case may be
%\begin{theorem} 
%\end{theorem}

% Uncomment and use as the case may be
%\begin{lemma} 
%\end{lemma}

%% The Appendices part is started with the command \appendix;
%% appendix sections are then done as normal sections
%% \appendix




\label{}

% To print the credit authorship contribution details
\printcredits

%% Loading bibliography style file
%\bibliographystyle{model1-num-names}
\bibliographystyle{cas-model2-names}

% Loading bibliography database
\bibliography{ref}

% Biography
\bio{}
% Here goes the biography details.
\endbio

%\bio{pic1}
% Here goes the biography details.
\endbio

\end{document}

