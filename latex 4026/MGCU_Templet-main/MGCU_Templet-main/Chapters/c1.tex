% Chapter Template

\chapter{Introduction} % Main chapter title

\label{c1} % Change X to a consecutive number; for referencing this chapter elsewhere, use \ref{ChapterX}

%----------------------------------------------------------------------------------------
%	SECTION 1
%----------------------------------------------------------------------------------------

\section{Introduction}
The ECG is widely used as a non-invasive diagnostic tool,  revealing useful information about the electrical activity of the heart. It's possible for healthcare professionals to learn a great deal about the cardiac health of individuals by analysing the temporal patterns and characteristics of ECG signals. Signal interpretation is a time-taking and intricate task,  therefore there is a chance of subjective ambiguity and human mistake in the analysis process,  even for experts who have spent years learning. As a result,  it is essential to prioritise computer-aided method for research  and development in time series analysis.  Computer-assisted analysis can analyse ECG signals more precisely and promptly,  with no differences caused by inter-operators or operator-specific differences \cite{liu2021deep}. Since ECG is a time series data,  it can be observed by work of \cite{dudukcu2023temporal} that hybrid deep neural network approaches outperform single deep neural network methods in dealing with time series prediction. The proposed model achieved an average RMSE of 0.0022 for chaotic data whereas 0.0082 for ECG Arrythmia dataset. Since, multiple model's integration outperform single traditionl model.   \cite{zhang2007neural} stated the proposed strategy consistently outperforms the single modelling approach for a range of time series processes. The analysis of electrocardiograms (ECGs) can be one of the areas in which it can be used extensively. PTB Diagnostic ECG dataset can be a valuable resource in this regard. A vast collection of ECG recordings is found in the PTB dataset,  including recordings from patients with a variety of cardiac conditions. Recordings capture the electrical activity of the heart over time,  creating a time series data structure. With the help of time series analysis techniques,  researchers and medical professionals can uncover hidden patterns,  identify abnormalities,  and develop predictive models for accurate cardiac diagnosis.
\par The use of automation in the field of cardiac illness diagnosis may aid in the accurate and timely examination of heart problems \cite{rahul2020exploratory}. The primary purpose of time series analysis of the PTB Diagnostic ECG dataset is to extract meaningful insights from the temporal dynamics of the ECG signals. Data preprocessing,  feature extraction,  anomaly detection,  and classification are some of the research areas widely performed in this domain. It is possible to detect specific ECG patterns associated with different cardiac conditions using advanced algorithms and statistical methods. This allows accurate diagnosis and timely treatment of the condition. ECG time series analysis as a means of improving cardiac diagnostics has the potential to revolutionise this field. In addition to enhancing detection accuracy,  it can also assist in the early detection of cardiovascular diseases,  and assist in the planning and implementation of personalised treatment plans. \par With our proposed model, It is possible,  too,  to use the knowledge gained from this analysis to develop automated ECG analysis systems,  empowering healthcare professionals with efficient tools for cardiac assessment. Using time series analysis techniques,  author aim to reveal the diagnostic potential of the PTB Diagnostic ECG dataset. To advance cardiac healthcare,  also seek to provide valuable insight into the temporal dynamics of ECG signals,  their association with cardiac conditions,  and their application to Cardiological treatment. (Note: The PTB Diagnostic ECG dataset refers to the publicly available dataset from the Physikalisch-Technische Bundesanstalt (PTB),  which contains ECG recordings of patients with different cardiac conditions).
