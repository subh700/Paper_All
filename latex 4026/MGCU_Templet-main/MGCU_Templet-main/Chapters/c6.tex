% Chapter Template

\chapter{Conclusion} % Main chapter title

\label{c6} % Change X to a consecutive number; for referencing this chapter elsewhere, use \ref{ChapterX}

\section{Conclusion}
Traditional DL models are implemented in this research by feeding PTB diagnostic ECG datasets from 5 different patients. These datasets are subjected to a time series analysis. A GRU-CNN hybrid model is constructed which is called proposed-1 throughout the studies. It has been seen that proposed-1 is outperforming traditional DL models. The significance of our investigation is based on the data that was collected. Our datasets' distinguishing qualities are highlighted. This research could help with future analyses of the PTB Diagnostic ECG database. Since, ECG is a one-dimensional signal,  1-D CNN is employed without any additional processing. However,  because CNNs learning is heavily reliant on data,  uneven data or inaccurate data labels can easily lead CNN models astray. The RMSE, MSE, MAE \& MAPE is shown to be giving least error on proposed-1(GRU-CNN) in comparison to the rest of the traditional DL models. To improve the results another model RB-GRU-CNN is proposed which is called proposed-2 throughout the studies. Proposed-2 has performed better than proposd-1 hence,  it can be concluded that GRU-CNN is better than traditional DL models and RB-GRU-CNN is better than GRU-CNN. Experimental validation for our proposed-1 and proposed-2 is done by non-parametric statistical analysis called Friedman test. 

\subsection*{CRediT authorship contribution statement}

\textbf{S.Khan} : Conceptualization, methodology, formal analysis, Resource, Writing original draft, visualization.\\
\textbf{Vipin Kumar} : methodology, conceptualization, validation, investigation, writing, review editing, supervision.

\subsection*{Data availability}



The data is available on the Physionet website. A full clinical summary is included in the header (.hea) file of the majority of these ECG records, including age, gender, diagnosis, and, when appropriate, data on medical history, medicines and treatment, coronary artery pathology, ventriculography, echocardiography, and hemodynamics. The clinical summary for 22 individuals is unavailable. This database has shown to be a valuable research resource for ECG investigations.





\subsection*{Acknowledgments}

A heartfelt applause to the Physionet's attempt in making this research possible by providing free clinical records to study cardiovascular diseases by people who have no prior idea about clinical sciences. 