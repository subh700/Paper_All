% Chapter Template

\chapter{Related work} % Main chapter title

\label{c2} % Change X to a consecutive number; for referencing this chapter elsewhere, use \ref{ChapterX}

%----------------------------------------------------------------------------------------
%	SECTION 1
%----------------------------------------------------------------------------------------

\section{Related work}
ECG data indicates the cardiac muscle's electrical activity as detected by electrodes put on the skin. The ECG is useful in detecting normal and acute coronary syndrome,  irregular heart rhythms,  and additional cardiac and noncardiac abnormalities. Researchers have also employed the ECG to analyse sleep,  emotions,  and stress. Several studies have found that physicians are frequently inept at reading ECGs in clinical practise \cite{bond2012effects}. Therefore a major number of academics have employed machine learning to see if AI can help enhance ECG interpretation and clinical decision making \cite{rjoob2022machine}. With DL,  we are attempting to introduce this approach of study. The main justification for employing DL The advantage of using it for ECG interpretation is that it may potentially analyse intricacies in ECG signals that clinicians do not normally investigate. One of the major reason behind using DL in this work is large number of data points. Since,  when the dataset is huge, DL algorithms can outperform other techniques \cite{purushotham2018benchmarking}. Also a vast majority of researches from air pollution \cite{kumar2022deep} to Medicine,  Automobiles \cite{naqvi2018deep},  Sales\cite{kaneko2016deep},  etc are being conducted in DL with promising results and future scopes.  Because of their capacity to extract features automatically,  deep neural networks have found remarkable success in a wide range of healthcare concerns. \cite{midani2023deeparr} gave a DeepArr model which provides an accuracy of 99.46\% on MIT-BIH Arrythmia dataset. A gated recurrent unit (GRU) was utilised to learn the intrinsic time features of SHM time series in order to extract the temporal feature vectors \cite{li2022hybrid}. Their CNN-GRU Heave attained RMSE of 1.3655 whereas Pitch gave an RMSE of 0.0350 . 
\par The objective of the \cite{prakarsha2022time} was to find a method for forecasting signals in time series with high accuracy using neural networks and beat the conventional method of adaptive filters in this regard. They claimed their approach may even eliminate the need to de-noise the signal before analysis,  as ANNs outperformed LMS filters by giving a forecasting accuracy of 95.72\%. Novel filtering method like Butterworth filtering is introduced in the studies of ECG. However,  the main purpose is to perform feature extraction which is done by DL models itself. When dealing with univariate time series,  traditional discriminant analysis techniques may be applied to leverage time series features such as autocorrelations,  periodogram coefficients,  wavelet features,  and so on. Several writers have proposed univariate time series discriminant analysis \cite{maharaj2007discrimination}. \cite{hammad2021automated} found that using a single feature resulted in poor performance when compared to combination of features. Also they recommended to use wavelet transform with CNN which can ensure a good accuracy even for small datasets. 
\par In recent years,  traditional ML algorithms have acquired unsatisfactory performance as a result of methods known as handmade approaches. Handcrafted feature engineering, in which a data scientist selects a set of variables with predictive powers,  is not required for DL,  as DL can conduct this filtering automatically. Experiments were carried out on CNN,  RNN,  GRU,  LSTM,  and BiLSTM, and a hybrid model combining 1D-CNN and GRU is created. CNNs are also utilised for one-dimensional data processing,  such as time series analysis \cite{sajjad2020novel} and our dataset is univariate and one-dimensional too. \par A feed forward neural network with a hierarchical structure is referred to as a convolutional neural network \cite{chua1998cnn}. CNN employs the weight-sharing principle as stated by \cite{qiu2018variety} which performs well on non-linear challenges such as time series prediction. CNN has grown popular in ECG feature extraction due to its capacity to learn favourable features from ECG input data in order to recognise patterns\cite{wang2023inter}. Because each neuron's layer in GRU effects the output at subsequent moments,  it can be utilised to characterise time series and handle gradient disappearance and gradient explosion difficulties in long sequence training \cite{dey2017gate}. Furthermore,  the GRU has two gating units that regulate how new information is integrated with old information and how much of the previous information is retained to calculate the new state,  resulting in significantly reduced calculation time throughout the feature extraction process \cite{yao2021interpretation}. The potential difference between electrodes placed on distinct parts of the body tissue is measured and recorded with an electrocardiograph or a vector electrocardiograph to produce an ECG. The aberrant activity of the heartbeat can thus be demonstrated\cite{liu2021deep}. The ECG can predict coronary heart disease. There have been studies that show ECG is useful in terms of predicting both short- and long-term results
. For individuals suffering from myocardial infarction,  for example,  the earlier the irregular cardiac rhythm is recognised,  the better the chances of avoiding life-threatening complications and recovering \cite{pollard2000acute}. Because ECG signals have considerable noise and complexity,  identifying specific disorders can be time-consuming and labor-intensive. Another issue is individual variation \cite{de2011weighted}. According to the researchers Because many signs and markers of cardiovascular disease can be found in physiological data other than ECG,  vital data such as respiratory rate and blood pressure can be utilised in conjunction with ECG to aid in diagnosis.
\cite{xu2018raim}. Occurance of sleep apnea can be diagnosed with DL forecasting of single lead ECG. A forecasting accuracy of 94.95\% is obtained using deep recurrent neural network \cite{bahrami2022deep}. Mainly Classification tasks are being performed on ECG dataset \cite{strodthoff2020deep}. Classifications involving wavelet transforms and encoding decoding\cite{mewada20232d}. Algorithms in DL are referred to as a 'black box',  which means there is no transparency and,  as a result,  no explanation is available to the users to provide some rationale as to what is going on inside the black box or why the DL algorithm produces an outcome in particular \cite{von2021transparency}. CNN-BiLSTM achieved an accuracy of 95\% on diagnosing atrial fibrillation\cite{aldughayfiq2023deep}. An avaerage RMSE of 0.082 is achieved by \cite{yoo2023restoration} after restoring missing signals using ensemble model. A comprehensive study can be found in the review literature \cite{musa2023systematic}
