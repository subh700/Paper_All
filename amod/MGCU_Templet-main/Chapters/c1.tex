% Chapter Template

\chapter{Introduction} % Main chapter title

\label{c1} % Change X to a consecutive number; for referencing this chapter elsewhere, use \ref{ChapterX}

%----------------------------------------------------------------------------------------
%	SECTION 1
%----------------------------------------------------------------------------------------

\section{Introduction}
Solar cells are energy devices that convert solar energy directly into electrical energy. Solar cells will account for 10\% of total energy consumption in 2030, implying that renewable energy sources will account for 30\%. Solar cells can account for up to 30\%  of Asian nations' energy resources when combined with wind turbines, thermal power plants, hydroelectric power plants, and power grid-integrated batteries. Significant reliability and management difficulties may occur with additional renewable energy to maintain the power grid system operational in the face of changing weather conditions, allowing the power grid to preserve renewable energy output while adjusting for deviations. Solar irradiance forecasts must be tailored to specific application requirements, ranging from short-term to long-term. Solar irradiance changes, such as ramp occurrences, must be predicted in extremely short-term and short-term time frames. These frequent fluctuations affect the dependability and performance of photovoltaic solar power systems. Short-term forecasting accuracy is critical for anticipating maximum solar power ramp rates, allowing for improved oscillation management and reduction \cite{brahma2020solar}.

However, it is vital to recognise the importance of medium and long-term forecasting horizons for enhancing operating planning and actively participating in the power system. This larger perspective allows for more effective planning and decision-making in the rugged terrain of energy management. Demonstrated how forecasting day-ahead solar irradiance might improve energy savings in a commercial building microgrid. The microgrid can optimise its energy consumption, storage, and distribution strategies by correctly anticipating solar irradiance a day in advance. To accomplish good solar energy forecasting, a suitable forecasting approach must be chosen based on the application's requirements. These approaches can range from statistical models to machine learning (ML) techniques, and their selection is influenced by factors such as prediction scope, data availability, and computational resources. The advantages of solar energy prediction may be improved while possible interruptions and inefficiencies in energy systems are mitigated by customising the forecasting technique to the application's unique requirements. DL algorithms have lately found traction in the prediction of solar irradiation. These models, a subset of ML methodologies, were designed to address complicated problems requiring enormous amounts of data. DL is well-known for its capacity to automatically extract precise properties from raw data, allowing crucial patterns to be detected\cite{husein2019day}.

As the amount of input data rises, DL models perform better than other types of models of ML models. The performance of regular machine learning models was compared to DL models when the amount of input data was changed. According to the study's findings, the predicting accuracy of DL models improves as the amount of training data grows\cite{rajagukguk2020review}. Models of traditional ML, on the other hand, tend to plateau once a given amount of data has been collected. DL's capacity to tackle complex forecasting problems such as solar irradiance prediction is evidenced by their efficacy growing as input data increases. DL models, designed to evaluate Time-series data, including text, audio, video, and photography, have significantly progressed and accomplished grand. RNNs, LSTM networks, GRUs, and hybrid models such as CNN-RNN and GRU-BiLSTM-LSTM are among the architectures covered by these models. These DL architectures performed admirably in sequential data-based solar forecasting models. LSTM and CNN models outperform Artificial Neural Networks (ANNs) and Support Vector Machines (SVMs) in forecasting short-term global horizontal irradiance (GHI), according to research. Because of their intrinsic capacity to examine and interpret precise temporal patterns, these models are valuable tools for addressing the significant difficulties of solar energy forecasting\cite{zang2020short}.

\subsection{In summary, the following are the work's major contributions:}
\begin{enumerate}
\item The research study targets India's eleven highly solar-irradiant cities for forecasting model learning.
\item  Two novel hybrid DL models have been proposed and evaluated for all eleven cities and stand-alone DL models for comparison purposes. 
\item The RSME and MAPE performance measures are used for quantitative comparision to demonstrate the effectiveness of proposed models over traditional models, along with graphical interpretation of effective performance. 
\item Furthermore, statistical analysis through Friedman ranking is also done to show the advantages of the proposed models. 
\end{enumerate}