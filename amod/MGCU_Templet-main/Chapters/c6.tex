% Chapter Template

\chapter{Conclusion} % Main chapter title

% \label{X} % Change X to a consecutive number; for referencing this chapter elsewhere, use \ref{ChapterX}

\section{Conclusion}
In this work, we studied alternate models for solar irradiance forecasts over various geographic regions. The RMSE was employed as a performance parameter to evaluate each model's accuracy. Models such as RNN, BiLSTM, LSTM, GRU, and CNN were investigated, as well as two suggested hybrid models, "Proposed1 (CNN-RNN)" and "Proposed2 (GRU-BiLSTM-LSTM)." The results reveal that model performance varies by location. Notably, in terms of performance, the suggested hybrid models outperform classic deep learning models. The following are the RMSE values for each location:
The GRU model has the lowest RMSE in Udaipur. The RMSE values in New Delhi vary among models, with LSTM and GRU performing best. Bhuj fared equally across models, with the lowest RMSE coming from the BILSTM model. Across all models, Bengaluru had the highest RMSE values, showing that estimating irradiance in this area is difficult. The model's performance varied in Hyderabad, Ahmedabad, Ahmedabad2, Pokhran, South Delhi, Ajmer, and Jodhpur, demonstrating the impact of local climatic conditions.
The forecasting model should be chosen considering the specific geographical area and meteorological factors. The suggested hybrid models, "Proposed1 (CNN-RNN)" and "Proposed2 (GRU-BiLSTM-LSTM)," have the potential to increase the accuracy of estimations of solar irradiation. Additional study and development of these models may increase their application in renewable energy management systems. This research illuminates the topic of solar energy forecasting, allowing for better-informed decisions on renewable energy integration and resource management.


\subsection*{CRediT authorship contribution statement}
\textbf{Amod Kumar:} Conceptualization, Methodology, Software. \textbf{Amod Kumar:} Data curation, Writing- Original draft preparation. \textbf{Amod Kumar:} Visualization, Investigation. \textbf{Vipin kumar:} Supervision. \textbf{Amod Kumar:} Software, Validation. \textbf{Amod Kumar and Vipin Kumar:} Writing- Reviewing and Editing.
\subsection*{Data availability}
The NASA POWER (Prediction of Worldwide Energy Resources) Data Access Viewer website is a massive and open-access library of meteorological, climatological, and renewable energy-related data that may be used by academics, scientists, and professionals all around the globe. The website, which has global geographic coverage, provides a wide range of information on periods ranging from hourly to monthly, including temperature, precipitation, wind speed, solar radiation, insolation, and more. Notably, it satisfies the needs of a wide range of research disciplines by distributing data in several formats and offering user-friendly interactive tools for data manipulation and visualisation. Its open access policy promotes transparency and cooperation, making it a dependable resource for climate studies, renewable energy appraisal, environmental research, and other purposes.
