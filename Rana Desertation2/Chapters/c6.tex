% Chapter Template

\chapter{Conclusion} % Main chapter title

\label{c6} % Change X to a consecutive number; for referencing this chapter elsewhere, use \ref{ChapterX}


% \section{Conclusion}
This research proposes a multiple bidirectional data transformation method by adding the positive (upward-shift) and negative (downward-shift) values over each training data point approach to build multiple learners using deep learning models to leverage the ensemble method. The UD-batch ensemble and CUD ensemble effectiveness have also been investigated with DL models, where UD-batch was an effective ensemble approach to overall measures. Temperature data from Delhi, India, has been utilised (i.e., downloaded from NASA website) and their statistical and decomposition exploratory analysis has been done after a careful preprocessing task.
The proposed MBiS-DT model's performance has been investigated using stand-alone DL models like BiLSTM, LSTM, GRU and RNN, where RSME, MAE, MAPE, and MSE are the performance measures utilised for the analysis. The analysis of the results over various measures shows that the proposed MBiS-DT, along with DL models, are outperformed, where the best performer among all traditional and proposed was MBiS-DT-BiLSTM. Further, the $R^2$ measure also shows the higher values of proposed models than the traditional models, which was also observed in the scatter plot of the original test and predicted test values. The effectiveness of the proposed models has also been analysed over non-parametric statistical analysis using Friedman ranking, where the MBiS-DT-BiLSTM model has shown the highest ranking among others. Pair-wise comparative analysis is also based on $p$-value ($\alpha=0.05$).

\subsection*{Declaration of conflicting interest:} The authors declare that they have no known competing financial interests or personal relationship that could have appeared to influence the work reported in this paper. 
\subsection*{Data Availability: } The data which is utilized in this study has been collected from the website of NASA with url \href{https://power.larc.nasa.gov/data-access-viewer/}{https://power.larc.nasa.gov/data-access-viewer} \cite{nasa}
%\section*{Declarations}

