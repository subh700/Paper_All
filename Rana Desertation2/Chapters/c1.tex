% Chapter Template

\chapter{Introduction} % Main chapter title

\label{c1} % Change X to a consecutive number; for referencing this chapter elsewhere, use \ref{ChapterX}

%----------------------------------------------------------------------------------------
%	SECTION 1
%----------------------------------------------------------------------------------------

% \section{Introduction}
Temperature prediction is a fundamental aspect of weather forecasting and climate research, essential for various applications across diverse domains, ranging from agriculture and energy management to public health planning and urban infrastructure development. Accurate temperature forecasts enable us to make informed decisions, proactively respond to extreme weather events, and adapt to changing climate patterns. While traditional statistical methods have been the cornerstone of weather prediction, recent advancements in DL have opened up new broadway for enhancing the accuracy and reliability of temperature forecasting. Likewise, impacts on food security \cite{darapaneni2021food}, human health \cite{miotto2018deep}, economy \cite{dehghani2021enhancing}, and energy consumption are anticipated \cite{cifuentes2020air}.

Several studies have investigated using LSTMs for climate modelling and weather forecasting. \cite{qin2017dual} The author developed an LSTM-based model for short-term precipitation forecasting, showcasing the potential of DL in capturing complex climate patterns. Ensemble learning strategies have been employed in tandem with DL models to enhance prediction accuracy\cite{shi2015convolutional}. DL, a subfield of artificial intelligence, has emerged as a robust tool for tackling such complex prediction tasks \cite{karpatne2017theory}. Its ability to automatically learn patterns and dependencies from data has revolutionized various domains, including meteorology. We are on a journey to track the power of DL for temperature prediction, focusing on time series data from Delhi, a region renowned for its climatic diversity.

Our primary objective is to explore the potential of Long Short-Term Memory (LSTM) networks, a specialized class of recurrent neural networks (RNNs), in modelling temperature time series. LSTM networks are adept at capturing long-range dependencies and are therefore well-suited for forecasting tasks involving sequential data

In this paper, we delve into applying DL techniques for temperature prediction using time series data from the bustling metropolis of Delhi, India. Delhi, the capital city and one of India's most populous urban centres, experiences diverse weather patterns influenced by natural climate variability and anthropogenic activities. Understanding and predicting temperature trends in this region is crucial for effective urban planning, resource management, and mitigating the impacts of heat waves and cold spells on public health and infrastructure. This paper embarks on a journey into DL, a revolutionary field of artificial intelligence that has exhibited remarkable prowess in modelling complex temporal data. By applying DL techniques to Delhi's time series temperature data, this research endeavour seeks to unlock valuable insights that can drive informed decision-making. The core motivation of this paper is rooted in the realization that climate science is far from linear; it embodies intricate non-linearities and dynamic interplays. This complexity makes neural networks, profound learning models, an ideal choice to capture the nuanced relationships within temperature time series data. With its capacity to unveil hidden patterns, DL is uniquely poised to discern the subtleties of climate systems and predict temperature fluctuations with unprecedented accuracy. Crucially, this paper not only explores the application of DL but also delves into the integration of advanced training algorithms. The fusion of backpropagation with genetic algorithms brings forth a robust and adaptable framework for model training, ensuring that our temperature predictions are not only accurate but also resilient to the complexities of climate systems. Air temperature prediction is a main climate factor essential for many applications in multiple areas such as tourism \cite{salman2015weather}, industry \cite{kumar2021opportunities}, agriculture \cite{mohan2018deep}, environment, energy, etc. \cite{abdel2004hourly}.

DL in temperature prediction is helpful due to its inherent capability to capture intricate temporal dependencies and nonlinear relationships within data. In the past few years, advanced machine learning models like Artificial Neural Networks (ANNs), LSTM networks\cite{wang2017predrnn} \cite{chen2021study}, and Convolutional Neural Networks (CNNs)\cite{chen2021correction} have shown great success across various areas. An example is the LSTM model, which was introduced by Sepp Hochreiter and Jiirgen Schrnidhuber\cite{graves2012long} 1997. LSTM is particularly good at handling data sequences like those in time series analysis. Among the DL architectures, we explore the Long Short-Term Memory (LSTM) network, celebrated for its capability in modelling sequential data. Our investigation extends beyond utilizing LSTM to introduce the Bi-LSTM, an innovative approach integrating local information for enhanced time series prediction.
Bi-LSTM offers a unique perspective by assigning higher importance to samples proximate to the test points, effectively capturing localized patterns critical for accurate forecasting. The methodology applied in this paper includes thorough data collection and preprocessing, which are essential for ensuring the integrity of our analysis. We get historical temperature data from reliable weather stations in Delhi from NASA's website, covering a specific period conducive to meaningful climate insights. Correct data cleaning tunes the missing values, outliers, and duplicates, while feature extraction results from the temperature-related attribute Temperature at 2 meters (T2M). Exploratory Data Analysis (EDA) is integral to our investigation, revealing underlying patterns, trends, and seasonality within the temperature time series data. Our visualizations and statistical analyses provide valuable insights into the data's temporal behaviour. The core of our research resides in DL model development, specifically LSTM, Bi-LSTM, GRU and RNN architectures \cite{tabrizi2021hourly}. We detail our model selection criteria, including hyperparameter tuning, to enhance prediction accuracy. The dataset undergoes partitioning into training testing sets, facilitating the model training and testing process. To ensure the reliability of our predictions, we conduct a thorough evaluation using performance metrics such as Root Mean Square Error (RMSE), Mean Squared Error(MSE), Mean Absolute Error (MAE), and Mean Absolute Percentage Error (MAPE).


This paper investigates DL models' effectiveness in predicting Delhi's temperature patterns. By utilizing historical temperature data collected from various meteorological stations, we seek to train and evaluate the performance of DL models and compare their results with traditional forecasting methods. The insights gained from this study will shed light on the strengths and limitations of profound learning-based temperature predictions and provide valuable guidance for future research and practical implementations.
\section*{The novelty of proposed research work:}

\begin{itemize}
\item A novel temperature data of Delhi, India, from NASA is utilized for this research work.
\item Multi-Bidirectional Shifting Based Data Transformation (MBiS-DT) is proposed for temperature prediction, which captures the different characteristics through a wise ensemble approach.
\item A suitable ensemble approach for MBiS-DT is analyzed among the Upward Batch and Downward Batch ensemble(UD-Batch ensemble) and the Corresponding upward and downward ensemble(CUD-ensemble).
\item MSE, MAE, MAPE, RMSE parameteters are utilized for comparision that finds the MBiS-DT-BiLSTM most suitable proposed model.
\item Statistical analysis through Friedman Ranking and p-values performed to distinguish the traditional model and proposed model performance.
\end{itemize}
The remainder of this paper is ordered as follows: Section 2 - provides a list of recent works, Previous work, and a literature review. Section 3 gives the DL models and performance measures. Section 4 includes steps of experiment and implementation setup and parameters settings. Section 5 presents the experimental results and their thorough analysis. The sixth section concluded with a review of existing and future research endeavours.
