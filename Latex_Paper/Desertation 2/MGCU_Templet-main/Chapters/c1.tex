% Chapter Template

\chapter{Introduction} % Main chapter title

\label{c1} % Change X to a consecutive number; for referencing this chapter elsewhere, use \ref{ChapterX}

%----------------------------------------------------------------------------------------
%	SECTION 1
%----------------------------------------------------------------------------------------

\section{Introduction}
Air pollution has become a major global problem due to industrialisation and urbanisation. The rising levels of air pollutants,  such as CO,  SO,  $O_3$,  $PM_{10}$ and $PM_{2.5}$. It has led to environmental issues like soil acidification,  fog, haze and severe health problems such as heart attacks and lung diseases. The World Health Organization has revealed that contaminated air affects most of the global population,  approximately 90\% \cite{zhou2019effects}. $PM_{2.5}$,  or 2.5 micrometres or less aerodynamic diameter particulate matter, is an essential factor in calculating the Air Quality Index (AQI). AQI is a numerical scale used to communicate how polluted the air is and its potential health effects to the public. $PM_{2.5}$ is a crucial air pollutant that can infiltrate the respiratory system and have detrimental health consequences. Time series data is a sequential data type collected regularly, with time as the index. It involves examining trends and patterns,  with stationarity important, implying statistical properties' constancy over time. Forecasting based on historical patterns finds widespread applications in various fields,  providing valuable decision-making and predictive modelling insights.
\par Three principal methods,  namely deterministic,  statistical,  and Machine learning (ML)/Deep Learning (DL),  are widely utilised in predicting air quality. Deterministic methods simulate atmospheric chemistry's dispersion and transport processes,  but they can be computationally expensive and less accurate due to limited actual observations. Statistical methods rely on historical data to forecast pollutant concentrations,  but their linear assumptions may limit prediction performance. Researchers are incorporating non-linear machine learning models to surpass these constraints as alternative methods for predicting air quality. Machine learning and Deep Learning models such as Support Vector Machines (SVM) \cite{lin2011forecasting}, Autoregressive-Integrated Moving Average (ARIMA) \cite{kumari2022machine},  Linear Regression(LR) \cite{kumari2022deep},  Artificial Neural Networks (ANNs) \cite{taylan2017modelling} and Fuzzy Logic (FL) \cite{wang2015model} have been applied in air quality prediction studies. ANNs have been particularly popular,  showing promising results in various applications. However,  the rapid development of deep learning techniques has outperformed traditional ML models. Deep learning models,  such as Long Short-Term Memory (LSTM) \cite{kristiani2022short},  Gated Recurrent Units (GRU),  and Convolutional Neural Networks (CNN) \cite{ayturan2018air},  have shown improved prediction performance by capturing long-term dependencies and spatial features in air quality data.
\par Multi-view learning \cite{zhao2017multi, xu2013survey} has emerged as a potent methodology in machine learning and deep learning. It effectively utilises multiple perspectives or representations of data to enhance predictive performance,  improve generalisation,  and address intricate real-world problems. The technique has garnered significant attention due to its ability to manage varied and complementary information from multiple sources or modalities. This approach holds promise in its application and usability for ML/DL models across different domains,  including Text and Image Analysis \cite{yang2020image, nie2017auto,YAN2021106, KUMAR2023101959,},  Audio and Video Processing \cite{garcia2018multi, hussain2021comprehensive, YAN2021106, KUMAR2023101959,},  and Environmental Monitoring \cite{huang2017multi}. Combining multi-view incorporation and hybrid deep learning models is a powerful technique in modern machine learning. This methodology improves predictive accuracy and feature extraction by integrating different data perspectives and utilising diverse neural network architectures. 
\par In time series analysis, the potential of multi-view learning has been shown by  \cite{9935292}, which has utilised multivariate heterogeneous features as multiple views such as geographic location and time of day. The author has compared the proposed multi-view multi-task (MVMT) model with a single-view dataset setting and shows better performance as an outcome. In another research \cite{atl2013mts}, a multi-view learning framework has been deployed for adaptive transfer learning to show the effectiveness of the inter-view usability of information transfer. In evaluating the proposed model, the time series classification task has been performed to get the generalised evaluations. Several multi-view learning for time series approaches corresponding to deep learning with time series \cite{mgtl2019tsc,vaw2019dtltsc}, machine learning with time series \cite{vaw2019dtltsc}, transfer learning \cite{mgtl2019tsc, 9935292}, etc. It has been observed from the literature that no research has been conducted yet based on a univariate time series dataset. Therefore, the potential of multi-view learning over univariate data may have an opportunity to perform time series prediction effectively.
\par In this research, the author has proposed a hybrid multi-views stacked CNN-BiLSTM model framework. The hybrid CNN-BiLSTM architecture has been utilised to build the two-stack network. The seasonal characteristics of the univariate data have been utilised to generate the views corresponding to the required number of views (called partitions of the data). Then, a stacked CNN-BiLSTM model was deployed over each data view for the predictions. After this, predictions of view models are ensembled to get the final prediction based on their validation performance at each view. The proposed model has been evaluated over seventeen univariate datasets of $PM_{2.5}$ pollutants and compared their effectiveness based on various performance measures.