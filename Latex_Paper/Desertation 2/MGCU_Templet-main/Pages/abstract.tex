%----------------------------------------------------------------------------------------
%	ABSTRACT PAGE
%----------------------------------------------------------------------------------------

\begin{abstract}
\addchaptertocentry{\abstractname} % Add the abstract to the table of contents
The presence of $PM_{2.5}$ is a significant concern for human well-being and ecosystems. The practical measure of $PM_{2.5}$ is the vital problem worldwide. These tiny particles can quickly enter the respiratory system and deeply infiltrate the lungs,  leading to various health issues,  including respiratory disorders,  cardiovascular diseases,  and premature death. The literature shows that hybrid deep learning (DL) models are performing better than stand-alone DL models of time series (i.e., CNN, RNN, GRU, LSTM and BiLSTM) to predict the $PM_{2.5}$ pollutant, but effective performance is not achieved yet. In this research, the author has proposed a hybrid stacked CNN Bidirectional-LSTM model architecture that utilises the multiple views of the data corresponding to seasonal repetitions to induce the multiple models, called Multi-view Stacked CNN Bidirectional-LSTM (MvS CNN-BiLSTM). The proposed model has been deployed over seventeen univariate time series ($PM_{2.5}$) data of highly polluted Indian cities and stand-alone DL models. The performances of the proposed model have been compared using Root Mean Square Error (RMSE) and Mean Absolute Percentage Error (MAPE) measures. The average enhancement of the proposed model on all datasets has been achieved compared to stand-alone DL models as RMSE:  7.11\% (CNN), 5.08\% (RNN), 3.80\% (GRU), 5.57\% (LSTM) and 4.05\% (BiLSTM) and MAPE: 27.16\% (CNN), 28.52\% (RNN), 26.22\% (GRU), 27.22\% (LSTM), 23.11\% (BiLSTM). Moreover, the non-parametric statistical analysis (Friedman and Holm\'s) have been performed and proves that the proposed model MvS CNN-BiLSTM is performed as distinct and compelling over both performance measures.
\end{abstract}
