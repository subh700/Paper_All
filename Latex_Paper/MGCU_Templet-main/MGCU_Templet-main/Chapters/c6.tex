% Chapter Template

\chapter{Conclusion} % Main chapter title

% \label{X} % Change X to a consecutive number; for referencing this chapter elsewhere, use \ref{ChapterX}

\section{Conclusion}

In this study, a hybrid MvS CNN-BiLSTM model has been proposed. The model utilises stacked 1D CNN \& BiLSTM as new architecture to view the univariate data $PM_{2.5}$ pollutant. Additionally, a novel multi-view approach has been proposed for univariate time series, which exploits the seasonal characteristics of the data to construct the views corresponding to the lowest lag. The seventeen datasets ($PM_{2.5}$) of highly polluted cities of India have been used to deploy the proposed MvS CNN-BiLSTM and stand-alone DL models. The evaluation of the models has been performed using RMSE and MAPE. The results have shown that the proposed model has improved the performed than corresponding stand-alone DL models over RMSE:  7.11\% (CNN), 5.08\% (RNN), 3.80\% (GRU), 5.57\% (LSTM) and 4.05\% (BiLSTM) and MAPE: 27.16\% (CNN), 28.52\% (RNN), 26.22\% (GRU), 27.22\% (LSTM), 23.11\% (BiLSTM). Moreover, the enhanced performance of the proposed model has also been validated using non-parametric statistical methods, i.e., Friedman ranking and Holm's procedure. The statistical analysis of results concludes that the proposed model performance is better and distinct from stand-alone DL models.

Future research: The proposed model has utilised two-stacked CNN-BiLSTM for this research, where multiple combinations of stand-alone DL models may be investigated along with the stacking of the hybrid for better performance. In a multi-view approach, apart from seasonal characteristics, trends and the remainder may be utilised to generate the views of the dataset, yielding enhanced performance. Moreover feature set partitioning method of multi-view learning may be utilised for multivariate time series data (single source or multi-source data).  

\subsection*{Data availability}
Data will be available on \href{https: //app.cpcbccr.com/ccr/#/caaqm-dashboard-all/caaqm-landing}{CPCB (India)} Webportel.The \href{https: //www.cpcb.nic.in/}{Central Pollution Control Board (CPCB) } in India maintains a web portal that offers access to various environmental datasets,  including air and water quality,  emission inventories,  and pollution monitoring data,  aimed at promoting environmental awareness and research in the country.

\subsection*{Acknowledgments}
We extend our heartfelt appreciation to the Central Pollution Control Board (CPCB),  India,  for generously providing invaluable environmental data,  which significantly enhanced our research and played a pivotal role in completing this study.
